% Document Type: PDFLaTeX
% Master File: bit.tex
\documentclass{article}
\usepackage{fullpage}
\usepackage{amsmath,mlsmath}

\begin{document}
\title{A Candidate Mathematical Formulation for Built-In-Test (BIT)}
\author{
  Karl {\AA}str\"{o}m \qquad
  George Bollas \qquad
  Benson Christalin \qquad
  Alberto Ferrari \\
  William Hale \qquad
  Clas Jacobson \qquad
  Marten Lohstroh \qquad
  Marco Marazza \\
  Yilin Mo \qquad
  Richard Murray \qquad
  Kyle Palmer \qquad
  Young Park \qquad
  Rich Poisson
}
\date{DRAFT. \today}
\maketitle

\section{Built-In-Test for a Single System}

Consider first a (abstract) dynamical system whose state propogates
according to the following equation:
\begin{displaymath}
  x' = f(x, \theta, u), \qquad x(t)\in\reals^n,\, \theta\in\reals^p,\,
  u(t) \in\reals^m,
\end{displaymath}
where $x(t)$ is the state, $\theta$ are a set of parameters and $u$ is a
(control) input.  We use $x'$ to represent the future state and
allow the possibility that it represents the time derivative $\dot x$,
a discrete-time update $x[k]$, or (discrete) state transition $x'$.

As a starting point, we assume that the system has a controller,
depending on the control mode $\alpha \in \integers$, of the
form $$u = k_\alpha(\hat x, r)$$ that maps the current (estimated) state of the
system along with a reference input (command).  We write this here as
a static mapping but this may be a more complex (dynamical) system,
such as a dynamic compensator (PID, MPC, etc).  The role of the mode
$\alpha$ is to allow that the system may operate in different discrete
modes that use different controllers.

A simple way of defining a ``fault'' is to say that a change in the
dynamics occurs that is reflected through a change in the values
describing the parameters $\theta$.  In this definition, we might
define a set of parameters $\Theta_{\text{fault},\alpha} \subset
\reals^p$ that corresponds to one or more fault conditions in contoller
mode $\alpha$.  We further define a set of fault codes that are
parameterized by an integer-valued multi-function $\text{FC}:\reals^p
\to 2^\integers$ where
\begin{displaymath}
  \text{FC}(\theta) = 
  \begin{cases}
    \emptyset, & \theta \notin \Theta_{\text{fault},\alpha}
      \quad\text{(no fault)}, \\
    \{\text{fc}_1, \dots, \text{fc}_k\},
      & \theta \in \Theta_{\text{fault},\alpha}
        \quad\text{(one or more faults)}.
  \end{cases}
\end{displaymath}
Note that $\text{FC}(\theta)$ can have multiple values, with each
value $\text{fc}_i$ corresponding to a fault code (there may be more
than one for a given system configuration).  We also explicitly keep
track of the fact that the subset of parameter values corresponding to
a given fault might depend on the controller mode $\alpha$.

Detecting faults depends on measurements of the system, which we
assume can be modeled in the form
\begin{displaymath}
  y = h(x, \theta) + w
\end{displaymath}
where $y \in \reals^l$ represents a vector of measurements and $w$
corresponds to some noise process.  We can either assume the sensors
are given ($h$ is specified) or that we are allowed to choose the
sensed varaiables ($h$ can be chosen from some family of possible
measurements).

\paragraph{Continuous BIT (CBIT)}
Continuous built-in-test measures the input and output signals $u(t),
y(t)$ and attempts to determine whether $\theta \in
\Theta_\text{faulty}$ and, if so, returns a set of fault codes
$\text{FC}(\theta)$.  If we don't measure $\theta$ directly, we might
use an estimate $\hat\theta$ (estimated from the input/output data)
that satisfies the property
\begin{displaymath}
  \text{FC}(\hat\theta) \subseteq \text{FC}(\theta)
  \qquad\text{or}\qquad
  \text{FC}(\hat\theta) \cap \text{FC}(\theta) \neq \emptyset.
\end{displaymath}
The first condition says that we get a set of fault codes that are a
subset of the actual fault codes; the second condition says that there
is at least one fault code is reported that matches the actual fault
codes.  In addition to a ``steady state'' fault determination, there
may also be performance requirements for how quickly the fault is
detected. 

\paragraph{Initiated BIT (IBIT)}
In initiated built-in-test, we allow for the possibility of commanding
new reference points $r$ that might help in the determination of the
underlying parameter values.  The corresponds roughly to the notion of
``forced response testing'', in which we choose a set of inputs that
allow us to better (or more quickly) identify the parameters.  Because
we are changing the reference to the system, initiated BIT is
typically use when the system is not in normal operation.  We assume
that no additonal equipment is needed to executed IBIT; hence the
original system dynamics and controllers are the only things available
to the IBIT algorithm.

\section{Built-In-Test for a System of Subsystems}


\end{document}